\documentclass[
	ruledheaders=section,
	class=report,
	thesis={type=bachelor},
	accentcolor=1c,
	custommargins=true,
	marginpar=false,
	%BCOR=5mm,%Bindekorrektur, falls notwendig
	parskip=half-,
	fontsize=11pt,
%	logofile=example-image, %Falls die Logo Dateien nicht vorliegen
]{tudapub}

%%%%%%%%%%%%%%%%%%%
%language correction
%%%%%%%%%%%%%%%%%%%
\usepackage[english]{babel}
\usepackage[autostyle]{csquotes}
\usepackage{microtype}

%%%%%%%%%%%%%%%%%%%
%Additional packages
%%%%%%%%%%%%%%%%%%%


%%%%%%%%%%%%%%%%%%%
%Bibliography
%%%%%%%%%%%%%%%%%%%
\usepackage{biblatex}   % Literaturverzeichnis
\bibliography{sample}


%%%%%%%%%%%%%%%%%%%
%packages tables
%%%%%%%%%%%%%%%%%%%
\usepackage{array}     % Basispaket für Tabellenkonfiguration, wird von den folgenden automatisch geladen
\usepackage{tabularx}   % Tabellen, die sich automatisch der Breite anpassen
\usepackage{longtable} % Mehrseitige Tabellen
\usepackage{xltabular} % Mehrseitige Tabellen mit anpassarer Breite
\usepackage{booktabs}   % Verbesserte Möglichkeiten für Tabellenlayout über horizontale Linien

%%%%%%%%%%%%%%%%%%%
%packages maths
%%%%%%%%%%%%%%%%%%%
\usepackage{mathtools} % erweiterte Fassung von amsmath
\usepackage{amssymb}   % erweiterter Zeichensatz
\usepackage{siunitx}   % Einheiten

%Formatierungen für Beispiele in diesem Dokument. Im Allgemeinen nicht notwendig!
\let\file\texttt
\let\code\texttt
\let\tbs\textbackslash

\usepackage{pifont}% Zapf-Dingbats Symbole
\newcommand*{\FeatureTrue}{\ding{52}}
\newcommand*{\FeatureFalse}{\ding{56}}

\begin{document}

\Metadata{
	title=Data-driven modeling of self-ignition properties of the renewable fuel PODE (Polyoxymethylen Dimethyl ether) using methods of machine learning,
	author=Pascal Roth
}

\title{Data-driven modeling of self-ignition properties of the renewable fuel PODE (Polyoxymethylen Dimethyl ether) using methods of machine learning}
\subtitle{Datenbasierte Modellierung des Selbstzündungsverhaltens des erneuerbaren Kraftstoffs OME (Oxymethylether) mit Methoden des maschinellen Lernens}
\author[P. Roth]{Pascal Roth}
\studentID{2541363}
\birthplace{Offenbach am Main}
\reviewer{Prof. Dr.-Ing. C. Hasse \and M. Sc. P. Haspel \and M. Sc. J. Bissantz}

\department{mb} 
\institute{Simulation reaktiver Thermo-Fluid Systeme}
%\group{Arbeitsgruppe}

\submissiondate{\today}
\examdate{\today}

%	\tuprints{urn=1234,printid=12345}
%	\dedication{Für alle, die \TeX{} nutzen.}

\maketitle

\affidavit

\tableofcontents


\chapter{Introduction}

\minisec{Polyoxymethylene Dimethyl ether}

\chapter{Reactor simulation and data computation}

\section{Homogeneous Reactor}

\section{Data compuation}

\chapter{Neural Network }

\section{Architecture}

\section{Training}

\chapter{Results and Discussion}

\section{Results of the Neural Network}

\section{Comparison to other models}

\minisec{Compartison to detailed mechanism}

\minisec{Comparison to Global Reaction Mechansim}

\begin{tabularx}{\linewidth}{@{}p{.25\linewidth}*3{>{\centering\arraybackslash}X}@{}}
	\toprule
	&&&\\
	\midrule
	&&&\\\midrule
	&&&\\
	\bottomrule
\end{tabularx}

\chapter{Conclusion}
\printbibliography

\end{document}
